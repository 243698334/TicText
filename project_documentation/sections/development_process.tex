% Make sure to address the issues of iterative development, refactoring, testing and collaborative development (even if you are not using XP, you have to address these issues in your documentation).

\subsection{Motivation}
% Introduction
In the software development world, every project ultimately descends into chaos. Teams proactively battle against this entropy using formal software development methods. These methods can slow a project's freefall, or when used effectively stall it entirely. One method we learned and applied in Software Engineering I was XP, or Extreme Programming. XP is a branch under the Agile Software Development methodology and derives many of its practices from Agile Software Development including fast iteration cycles, use cases, and user stories; but, XP differs from typical Agile Software Development methodology through pair programming and complete unit testing. 

% Pair Programming
While most groups in Software Engineering I followed many XP practices, fewer groups adhered strictly to pair programming. Why were many students repulsed by the idea of pair programming when the remaining groups reported success with it? Could group dynamic, individual skill, an unnecessary byproduct of our interspersed and erratic location on campus at any given time, or something else determine pair programming’s necessity and success? We believe that all of the aforementioned concerns contribute in peer hesitation to adopt and consistently practice pair programming. Some group atmospheres do not support or encourage their members to admit unfamiliarity or inexperience. Other individuals feel their skill does not necessitate someone else watching over their shoulder as they crank out clean code–line by line. And the sheer size and scatterness of the campus make it difficult for students to easily coordinate a place and time for pair programming. 

% Testing
In software development, investors and stakeholders want to see visible results and often pressure developers to focus on functionality. To please the stakeholders, beginner developers will write code that only meet the requirements. In school and personal projects, this procedure is often adequate to achieve full marks or a deployable product, but quickly deteriorates in a collaborative and large-scale projects. In such projects, new code is introduced and old code is refactored so frequently that it's inevitable for a code defect to appear. Under Agile Software Development methodology, development teams simultaneously use unit testing to verify code being committed and to guard against future refactors and design changes.

% Version Control
To improve team efficiency and facilitate grading, the University of Illinois instructors require students to use Subversion: a program (more accurately “a lifestyle choice”) that helps software developers maintain and track all changes to, or version, a code repository. An alternative versional control system to Subversion is Git. The biggest difference between Git and Subversion is how collaboration is organized. Subversion uses centralized remote repositories, which means that each collaborator syncs every change with a single master repository over the Internet. Git, on the other hand, uses decentralized repositories. In decentralized repositories, there is no single master repository and each collaborator’s local repository is autonomous. Because each local repository is autonomous, collaborators can reliably commit changes to their local repository even without an Internet connection. These changes can then be, when an Internet connection becomes available, pushed and pulled to and from other collaborators. This workflow makes Git’s reliability far superior to Subversion. Both Subversion and Git let collaborators create branches to pivot development for a specific feature, but Git further augments collaboration through a different feature: repository forks. When a collaborator forks a repository, they create an isolated copy of the original repository. In this new isolated repository, collaborators can create, delete, and commit changes to branches without worrying about conflicting with another collaborator. In case something goes awry, projects using Git commonly designate a stable branch on the main repository to guarantee a safe point to revert. For a project that wants to maximize efficiency, Git offers our project unparalleled autonomy, reliability, and safety.

% Code Review/Iterative Development
While each collaborator using Git has a local copy of the main repository, many projects backup their repository to a remote service, such as Github, and designate their repository on Github as their main repository. Don’t be fooled–Github is more than just a service to host a remote Git repository; its website provides many tools to enhance efficiency.  One of these tools is the pull request: a proven, efficient, and convenient practice to submitting, review, and merge collections of changes. Pull requests are similar to formal code reviews, a process software development teams use to ensure code quality in incoming changes. Prior to completing a feature, the submitter submits their changes for a formal code review where they receive critique and feedback from other developers. Often a feature must go through many iterations in this stage before becoming “production ready.”

\subsection{Our process}
% Pair Programming
Because of the reality of resistance to pair programming, we have adopted it on an as-needed basis. We hope that this compromise will let skilled individuals maintain their efficiency while also helping the bottom line–those that may not be experts in the technology.

% Testing
Our project uses XCode's native XCTest framework for unit testing, OCMock for comprehensive testing, and Automation for interface testing.

% Version Control
Our project hopes to improve efficiency adjustments to our project’s software development method by versioning our code with Git instead of Subversion.

% Code Review/Iterative Development
While Github doesn’t natively facilitate formal code reviews, we self-impose a rule that each pull request must be reviewed and approved by two other team members before merging. Reviewers can use Github’s web interface to review and comment on the pull request, its files, and even individual lines of code. 

% Refactoring
Following Agile Software Development methodology, we are always refactoring the codebase. We refactor while coding, while writing tests, and while revisiting code when implementing a new feature.

% User Stories
But before we can even write a line of code, we need to know what features we want and when we want them. To solve this problem, we create user stories and assign each story a story point value. We hope to relieve some of the stress and inherent instability with assigning a fixed man-hour value while still providing an accurate estimate to facilitate the planning and management of iterations. Through user stories, we hope to be able to accurately portray the needs of the users while also providing realistic time estimates to help us pace ourselves so that we may finish the project within the deadlines.