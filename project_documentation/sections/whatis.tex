%% Brief Description
\subsection{Brief Description}
TicText is an iOS messaging app that allows you to send expire-able text with your friends. We call this kind of text "Tic", as in the sound of a clock counting down (Tic Tok, Tic Tok...). For each Tic sent to your friends, you will be able to set a time duration, which will be the amount of time before the Tic got expired and becomes unable for your friend to retrieve its content. That is to say, the recipient can only retrieve the content from our server within a certain time period after the sender sent the Tic. Here is an example of a possible scenario of using TicText:

\begin{dialogue}
	\speak{John} \direct{\refer{He} is drunk at 3am and sends a tic using \refer{His Phone} to \refer{Mary} with a 60 minute expiration time. The tic reads...}
	
		\medskip
		Hey babe, I miss you so much and I really regret breaking up with you...
	
		\medskip
		\direct{\refer{Mary} (while sleeping) receives a notification on her phone.}
	\speak{Mary's Phone} Someone has just sent you a Tic! Swipe to read before it's too late. Tic Tok...
	
		\medskip
		\direct{The next morning, \refer{Mary} saw the notification and opened the app. The only thing left was...}
	\speak{TicText on Mary's Phone} Sorry you've missed a Tic which expired on 4:00 am.
\end{dialogue}

TicText would also let you to send a Tic anonymously - even if the recipient saw the notification in time, he/she still won't be able to see the sender's name.

We believe TicText is also capable of creating romance in many ways. But in the example above, well, hopefully TicText just avoid something awkward or even more. 
%% Motivation
\subsection{Motivation}
We would like to build something brings people joy and more importantly, works even with limited amount of users. So we gave up on the ideas related to sharing, social networking, or dating, and focused on single user or one-on-one experiences. We believe TicText could be something we, especially the younger generation, would enjoy. Also, building a mobile app with modern technologies will always be challenging and exciting. We're sure we could learn a lot and have fun as well. 

%% Risks/Challenges
\subsection{Risks/Challenges}
One of the biggest challenges would be building a good-looking UI with great user experience. There aren't many features in TicText, so User Experience is a top priority. In the meantime, we have to think about the security of all Tics as well. The last thing we want is to compromise a user's privacy. In order to do that, we need more than beautiful code; we also need to code in the "right way" using the "right tools.''