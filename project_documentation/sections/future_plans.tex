\begin{table}[h]
	\centering
	\caption{Future Stories}
	 \renewcommand{\arraystretch}{1.2}
	\rowcolors{2}{white}{gray!20}
	\begin{tabular}{>{\centering\arraybackslash}m{2.5cm} | m{11.5cm} }
		\toprule
		Story Points & Description\\
		\midrule
		5 	& As a user, I want to send a voice message Tics to a TicText friend.\\
		3 	& As a user, I want to be able to set the app's color scheme in the in-app Settings\\
		5 	& As a user, I want to be able to set and sync a background image in a conversation (MsgVC) with my friend\\
		8 	& As a user, I want to be able to enter my phone number and have the app use my Address Book to automatically match us.\\
		3 	& As a user, I want to be able to check someone's profile from inside the conversation view.\\
		3 	& As a user, I want to be able to send a text with an invitation to join TicText to a person in my Address Book\\
		\bottomrule
	\end{tabular}
\end{table}

Here are some personal reflections on the project and process by current or former team members:

% Terrence Katzenbaer
\begin{quote}
\lipsum[2]
\attrib{Terrence Katzenbaer}
\end{quote}

% Kevin Chen
\begin{quote}
\lipsum[2]
\attrib{Kevin Chen}
\end{quote}

% Chengkan Huang
\begin{quote}
I learned a lot by working on this project, in terms of both iOS development skill and software engineering technique. Before this semester, I have never done any iOS development. Now, by learning from my experienced team members and online sources, I am more proficient in this field. I wrote code following consistent code convention with documentation and tests. We followed XP, always creating pull request before merging into master and reviewing/commenting others’ code. And we have user stories done after every iteration.
\attrib{Chengkan Huang}
\end{quote}

% Jack Arendt
\begin{quote}
I learned how to incorporate my project into parse and use different testing frameworks. I worked on the FindFriendsViewController and the original conversation view. I also worked on creating the header above the settings view controller. I liked using parse, I thought it was a great framework, I enjoyed the code reviews, it made me a much better coder and more careful of what I was writing. Having branches that weren't merged into master after iterations ended up causing a lot of headaches down the road.
\attrib{Jack Arendt}
\end{quote}

% Georgy Petrukhov
\begin{quote}
While working on this project I learnt a lot about iOS development and about Parse framework. I found it useful to meet each week to discuss our work and plan for the iteration meeting with the TA (who is simulating a client). That definitely helped everyone to be on track with the current state work. Also using Github and its Pull Request system made code review possible. Everyone who has access to the repository can review the history of commits, which makes our work very transparent to an outsider.
\attrib{Georgy Petukhov}
\end{quote}
